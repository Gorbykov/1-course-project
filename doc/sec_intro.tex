\sectioncentered*{Введение}
\addcontentsline{toc}{section}{Введение}
\label{sec:intro}

В век информационных технологий появляются огромные массивы данных, которые можно и нужно уметь обрабатывать с помощью вычислительной техники с целью извлечения знаний.
Статистическое моделирование и интеллектуальный анализ данных представляют необходимые инструменты и способы обработки и анализа больших объемов данных.
В данном дипломном проекте рассматривается один из способов информационно"=статистического  моделирования "--- вероятностные сети, в частности, байесовы сети доверия.

Байесовы сети применяются для решения различных практических задач.
Вероятностная природа сетей способствует их успешному применению для создания различных экспертных систем.
Одним из первых практических проектов, использующих байесовы сети, стала система медицинской диагностики PathFinder-4~\cite{terehov_2003}.
В прикладном программном обеспечении байесовы сети используются в различных пошаговых мастерах по диагностике неисправностей, исправлению ошибок, консультированию пользователей, например, диагностика неисправности оборудования в Windows~\cite{terehov_2003}.
Вероятностные сети применимы также для создания рекомендательных систем, основанных на предпочтениях пользователя и истории его активности.

Важным этапом в применении вероятностных сетей для решения какой"=либо задачи является, собственно, её построение, а именно задание структуры сети.
Под структурой понимается задание отношений независимости между парами вершин, которые соответствуют случайным величинам из исходной задачи.

Исторически одним из первых способов построения структуры байесовых сетей было привлечение экспертов в конкретной предметной области и разработка архитектуры сети в соответствии с представлением экспертов о решаемой задаче и предметной области.
Данный способ имеет ряд очевидных недостатков: необходимость привлечения экспертов; большая трудоемкость процесса построения сети для сложных задач с большим количеством случайных величин и, соответственно, большим количеством узлов; ограниченность модели представлением эксперта о задаче и предметной области.

Появляются новые предметные области и классы задач, в которых довольно сложно найти признанного эксперта.
В такой ситуации становится понятным, что привлечение эксперта для разработки байесовой сети не всегда возможно и расточительно по времени.
С другой стороны, сбор экспериментальных данных для решения какой"=либо задачи обычно легко доступен.
В связи с этим возникает задача обработки этих данных для решения задачи.

В данном дипломном проекте рассматривается задача вывода структуры вероятностной сети по набору экспериментальных данных.
С учетом приведённых выше утверждений про потенциальную невозможность привлечения экспертов и доступность экспериментальных данных, умение строить сеть лишь по набору экспериментальных данных становится очень привлекательным.
Помимо ускорения процесса построения сети, автоматический вывод структуры имеет ряд дополнительных преимуществ перед <<ручным>>: становится возможным выявление ранее неизвестных зависимостей между переменными в известных и новых предметных областях; появляется возможность довольно легко обновлять структуру при получении более достоверных экспериментальных данных; создание и применение вероятностных сетей становится более доступным для не"=экспертов, и появляется возможность использования вероятностного подхода для решения б\'{о}льшего множества прикладных задач.

Не смотря на привлекательность автоматического построения структуры сети, вычислительно эта задача является $\mathcal{NP}$-полной~\cite{Chickering96learningbayesian}.
Многие существующие алгоритмы состоят из двух компонентов\ignore{%
В математике принято считать слово "компонента" женского рода, соответственно мн.ч. р.п будет "компонент", в других областях - компонент, сущ. м.р., во мн.ч. р.п. - компонентов. Источник http://bit.ly/10PMCfI и http://bit.ly/11Rsj85}: функции для оценки качества сети для имеющихся экспериментальных данных и процедуры поиска структуры сети, оптимизирующую выбранную оценочную функцию.
Во многих алгоритмах точное вычисление оценочной функции имеет экспоненциальную сложность по времени, но на практике с помощью различных допущений и оптимизаций её можно аппроксимировать за приемлемое время.
Пространство же возможных направленных ациклических графов имеет супер"=экспоненциальный порядок роста~\cite{robinson_1977}, и полный перебор в таком пространстве возможных решений на практике не возможен для задач с более чем семью наблюдаемыми случайными величинами.

На практике применяют несколько различных способов уменьшения пространства возможных решений, некоторые из них предполагают проведение предварительных вычислений для извлечения первичной информации о взаимоотношениях переменных, другие "--- требуют априорных знаний об исходном распределении и частичных знаний о зависимостях переменных.
Естественно подобные ухищрения в различной степени влияют на качество получаемого результата не в лучшую сторону, но зато позволяют существенно сократить пространство поиска, что в свою очередь позволяет в разумное время найти структуру сети, которая будет аппроксимировать <<истинное>> распределение из которого были получены экспериментальные данные.

В данном дипломном проекте реализуются некоторые из известных алгоритмов автоматического вывода структуры сети по данным, дополнительно производятся некоторые оптимизации и улучшения в части их реализации. 
Также были произведены экспериментальные модификации этих алгоритмов для улучшения качества выводимой сети. 
В результате получилась библиотека классов для платформы \dotnet{}, написанная на языках программирования \csharp{} и \fsharp{}, пригодная для решения практических задач в реальных проектах.
На данный момент в библиотеке реализована лишь возможность построить структуру сети по данным, но не затронуты очень важные и интересные вопросы, такие как обучение параметров сети и задача статистического вывода суждений в обученной сети.
